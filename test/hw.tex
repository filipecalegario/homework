%% hw.tex Copyright (C) 2020  Ahmad Tashfeen
%% This program comes with ABSOLUTELY NO WARRANTY.
%% This is free software, and you are welcome to redistribute it
%% under certain conditions; read file COPYING for more details.

\documentclass{../src/homework}
% hwlst & hwsymb  are not pointed to the ../src/

\author{Musa Al`Khwarizmi}
\class{CS 3141: Prof. Kamil's Algorithm Analysis}
\date{\today}
\title{Homework Class Test}
\address{Bayt El-Hikmah}

% tex.stackexchange.com/q/139401/215221
\graphicspath{{../media/}}
\usepackage{lipsum, hyperref}

\begin{document} \maketitle

\question What is this document?

This is a demonstration of my homework \LaTeX{} class. It is an extension of the \texttt{amsart} and should have all of its functionality. These are some of the set symbols: $\bC \supset \bR \supset \bQ \supset \bZ \supset \bN$, then some Greek and other mathematical symbols are, $\al, \ep, \p, \ra, \Ra, \injective, \surjective, \bijective$. We can also insert multiple figures as seen in figure \ref{trio}. There is also an \texttt{org-mode} version of this file \footnote{Tashfeen's \texttt{org-mode} configurations can be found \href{https://github.com/simurgh9/emacs786}{here}.}.

\fig[0.2]{khwarizmi.png, kitab.jpg, page.png}{Al`Khwarizmi}{trio}

\question Prove that $\exists (x,y) \in \bZ$ such that $x+y = 4$.
\begin{proof} Four is the sum of two integers.
  
  $1,3 \in \bZ$ and $1+3=4$.
\end{proof}

% \rightfig[0.15]{khwarizmi.png}{}{right_image}

\question\label{cardinality} What is the cardinality of Natural Numbers?

It is $\aleph_0$ \cite{arlinghaus1996part}. See also question \ref{custom-index}.

\question[99]\label{custom-index} Is the cardinality of Naturals and Reals the same because they are both infinite?

No, the cardinality of Reals is greater because they are also un-listable (uncountable). See also question \ref{cardinality}.


\question Finally the numbered bullets are done with the \texttt{enumitem} package,

\begin{enumerate}
  \item With just bullets,
  \begin{itemize}
    \item \bold{Cats}
    \item \italic{Dogs}
  \end{itemize}
\end{enumerate}

\begin{bonus} State chain rule.

  Chain Rule:
  \[
    \zeta(x) = f(g(x)) \quad \text{ then according to the chain rule: } \quad
    \derivative{\zeta} = \derivative[g]{f} \times \derivative{g}
  \]
\end{bonus}

\newpage

\begin{bonus} Euclidean Algorithm

  You may write code as in listing \ref{gcd},
  % https://en.wikibooks.org/wiki/LaTeX/Source_Code_Listings
  \lstinputlisting[language=Python, caption={Euclidean Algorithm for Greatest Common Factor},label=gcd]{../code/sample_code.py}
  
\end{bonus}

% \begin{prb}{77} Deprecated Custom Numbering Environment
%   Don't use \texttt{prb} environment.
% \end{prb}

% Tashfeen should run C-c C-s XeLatex BibTex XeLatex XeLatex
\bibliography{citations} 
\bibliographystyle{plain}

\end{document}