%% hw.tex Copyright (C) 2020  Ahmad Tashfeen
%% This program comes with ABSOLUTELY NO WARRANTY.
%% This is free software, and you are welcome to redistribute it
%% under certain conditions; read file COPYING for more details.

\documentclass{homework}

\author{Musa Al`Khwarizmi}
\class{CS 3141: Prof. Kamil's Algorithm Analysis}
\date{\today}
\title{Homework Class Test}
\address{Oklahoma State University}

% https://tex.stackexchange.com/questions/139401/how-to-use-graphicspath
\graphicspath{{media/}}
\lstset{language=python}
\usepackage{lipsum}

\begin{document} \maketitle

This is a demonstration of my homework \LaTeX{} class. It is an extension of the \texttt{amsart} and should have all of its functionality. These are some of the set symbols: $\bC \supset \bR \supset \bQ \supset \bZ \supset \bN$, then some Greek and other mathematical symbols are, $\al, \ep, \p, \ra, \Ra, \injective, \surjective, \bijective$. We can also insert multiple figures as seen in figure \ref{trio}.

\fig[0.2]{khwarizmi.png, kitab.jpg, page.png}{Al`Khwarizmi}{trio}

\begin{question} Prove that $\exists (x,y) \in \bZ$ such that $x+y = 4$.

  We show,
  \begin{proof} Four is the sum of two integers.
    
    $1,3 \in \bZ$ and $1+3=4$.
  \end{proof}
\end{question}

% \rightfig[0.15]{khwarizmi.png}{}{right_image}

\begin{bonus} Bonus Question Statement 

  \lipsum[2]
  \[
    \zeta(x) = f(g(x)) \quad \text{ then according to the chain rule: } \quad
    \derivative{\zeta} = \derivative[g]{f} \times \derivative{g}
  \]
\end{bonus}

\begin{bonus} Euclidean Algorithm

  You may write code,
  % https://en.wikibooks.org/wiki/LaTeX/Source_Code_Listings
  \lstinputlisting[language=Python]{./code/sample_code.py}
  
\end{bonus}

\begin{question} What is the cardinality of Natural Numbers?
    
  It is $\aleph_0$.
\end{question}

\begin{question} Is the cardinality of Naturals and Reals the same because they are both infinite?

  No, the cardinality of Reals is greater because they are also un-listable (uncountable).
\end{question}

\begin{question}[99] Custom Numbering.

  This question is numbered 99.
\end{question}

\begin{question}
Finally the numbered bullets are done with the \texttt{enumerate} package,

\begin{enumerate}
  \item With just bullets,
  \begin{itemize}
    \item \bold{Cats}
    \item \italic{Dogs}
  \end{itemize}
\end{enumerate}
\end{question}

% \begin{prb}{77} Deprecated Custom Numbering Environment
%   Don't use \texttt{begin\{prb\}}.
% \end{prb}

\end{document}